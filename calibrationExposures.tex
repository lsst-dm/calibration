%
% This file contains details of how many exposures we need to establish and maintain calibration
% products for the main telescope
%
\section{Proposed Calibration Exposures}
\label{sec:CalibrationExposures}

\begingroup                             % macros
\newcommand{\DspotProjector}{D_{\mbox{\tiny spot}}} % Diameter of spot projector
\newcommand{\OverfillFactor}{f_D}                   % factor by which we exceed 1.8m/\DspotProjector
\newcommand{\DlambdaFilter}{\Delta_\lambda}         % step in lambda for spots when a filter is in the beam
\newcommand{\DlambdaFlat}{\Delta_{\lambda,\mbox{\tiny flat}}}   % step in lambda for flats (no filter)
\newcommand{\DlambdaOffset}{\Delta_{\lambda,\mbox{\tiny hole}}} % step in lambda for spots when measuring hole sizes
\newcommand{\DlambdaWhite}{\Delta_{\lambda,\mbox{\tiny white}}} % step in lambda without a filter
\newcommand{\Nband}{N_b}                                        % number of filters
\newcommand{\NPointPerFilter}{N_{\lambda/\mbox{\tiny filter}}}  % number of samples/filter
\newcommand{\NspotCCD}{N_{\mbox{\tiny CCD}}}                    % number of spots per CCD
\newcommand{\Noffset}{N_{\mbox{\tiny offset}}}                  % number of offsets to solve for hole sizes
\newcommand{\NFlat}{N_{\mbox{\tiny flat}}}                      % number of exposures per flat
\newcommand{\Tdark}{t_{\mbox{\tiny dark}}}                      % elapsed time per set of dark data
\newcommand{\Tflat}{t_{\mbox{\tiny flat}}}                      % elapsed time per set of flat data
\newcommand{\Tspot}{t_{\mbox{\tiny spot}}}                      % elapsed time per set of spot data
\newcommand{\FracBandwidth}{f_{\mbox{\tiny bandwidth}}}         % fraction of total bandwith scanned per filter

\newcommand{\DspotProjectorNominal}{0.3\mbox{m}}
\newcommand{\OverfillFactorNominal}{2}
\newcommand{\DlambdaFilterNominal}{1\mbox{\tiny nm}}
\newcommand{\DlambdaFlatNominal}{100\mbox{\tiny nm}}
\newcommand{\DlambdaOffsetNominal}{100\mbox{\tiny nm}}
\newcommand{\DlambdaWhiteNominal}{10\mbox{\tiny nm}}
\newcommand{\NbandNominal}{6}
\newcommand{\NPointPerFilterNominal}{2}
\newcommand{\NspotCCDNominal}{5}
\newcommand{\NoffsetNominal}{9}         % not 10 as we can reuse one of the NspotCCD
\newcommand{\NFlatNominal}{10}
\newcommand{\TdarkNominal}{1000\mbox{\tiny s}}
\newcommand{\TflatNominal}{13\mbox{\tiny s}}
\newcommand{\TspotNominal}{45\mbox{\tiny s}}
\newcommand{\FracBandwidthNominal}{0.5}

\newcommand{\dspotProjector}{\left(\frac{\DspotProjector}{\DspotProjectorNominal}\right)}
\newcommand{\overfillFactor}{\left(\frac{\OverfillFactor}{\OverfillFactorNominal}\right)}
\newcommand{\dlambdaFilter}{\left(\frac{\DlambdaFilter}{\DlambdaFilterNominal}\right)}
\newcommand{\dlambdaFlat}{\left(\frac{\DlambdaFlat}{\DlambdaFlatNominal}\right)}
\newcommand{\dlambdaOffset}{\left(\frac{\DlambdaOffset}{\DlambdaOffsetNominal}\right)}
\newcommand{\dlambdaWhite}{\left(\frac{\DlambdaWhite}{\DlambdaWhiteNominal}\right)}
\newcommand{\nband}{\left(\frac{\Nband}{\NbandNominal}\right)}
\newcommand{\nPointPerFilter}{\left(\frac{\NPointPerFilter}{\NPointPerFilterNominal}\right)}
\newcommand{\nspotCCD}{\left(\frac{\NspotCCD}{\NspotCCDNominal}\right)}
\newcommand{\noffset}{\left(\frac{\Noffset}{\NoffsetNominal}\right)}
\newcommand{\nFlat}{\left(\frac{\NFlat}{\NFlatNominal}\right)}
\newcommand{\tdark}{\left(\frac{\Tdark}{\TdarkNominal}\right)}
\newcommand{\tflat}{\left(\frac{\Tflat}{\TflatNominal}\right)}
\newcommand{\tspot}{\left(\frac{\Tspot}{\TspotNominal}\right)}
\newcommand{\fracBandwidth}{\left(\frac{\FracBandwidth}{\FracBandwidthNominal}\right)}

\begin{table}
\begin{center}
  \begin{tabular}{lllp{4cm}}
    What & Symbol & Nominal & Notes \\
    \hline
    Radial extent of M1 & & 1.8m & $4.2m - 2.4m$ \\
    Number of bands & $\Nband$ & $\NbandNominal$ \\
    Total LSST bandwidth  & & 700nm & \\
    \noalign{\medskip}
    Fraction of bandwidth scanned for each filter & $\FracBandwidth$ & $\FracBandwidthNominal$ \\
    Diameter of spot projector & $\DspotProjector$ & $\DspotProjectorNominal$ \\
    Oversampling of M1 & $\OverfillFactor$ & $\OverfillFactorNominal$ & $\OverfillFactor = 88$ to tile M1 \\
    Step in $\lambda$ for spots (with a filter) & $\DlambdaFilter$ & $\DlambdaFilterNominal$ \\
    Step in $\lambda$ for spots (without a filter) & $\DlambdaWhite$ & $\DlambdaWhiteNominal$ \\
    Step in $\lambda$ when measuring mask throughput & $\DlambdaOffset$ & $\DlambdaOffsetNominal$ \\
    Step in $\lambda$ for flats (without a filter) & $\DlambdaFlat$ & $\DlambdaFlatNominal$ \\
    Number of spots per CCD & $\NspotCCD$ & $\NspotCCDNominal$ \\
    Number of offsets to solve for hole sizes & $\Noffset$ & $\NoffsetNominal$ & can use one of spectral scans; total 10 \\
    Number of samples/filter for contamination check & $\NPointPerFilter$ & $\NPointPerFilterNominal$ \\
    Number of exposures per flat & $\NFlat$ & $\NFlatNominal$ & per-pixel S/N $\sim$ 1000 \\
    Elapsed time per set of flat data & $\Tflat$ & $\TflatNominal$ & 10s exposure \\
    Elapsed time per set of spot data & $\Tspot$ & $\TspotNominal$ &
    Assume 4s overhead, 15s exposure, and \c 26s overhead to move telescope/projector \\
  \end{tabular}
  \caption{
    Parameters involved in defining calibration sequences
  }
  \label{tab:CalibrationParams}
\end{center}
\end{table}

Nominal values for parameters such as the spot projector diameter, wavelength steps, and exposure times are
given in Table \ref{tab:CalibrationParams}.  This table does \textit{not} include filter change times, which
there need to be added and should add at most 10--15 minutes per day (not trivial).

Let us initially assume that at all times we know the filter bandpasses $\qe_b(\lambda, \xb)$
as a function of position on the filter and wavelength.  We'll consider whether we will be able
to measure the bandpasses using the flat fields and spot projector in Section \ref{sec:MeasureFilters}.

%------------------------------------------------------------------------------
%
% This file is created by running the exposureTable.ipynb notebook
%
\import{.}{exposureTable}

%------------------------------------------------------------------------------

\subsection{Required Number of Exposures}

The number of exposures and elapsed time resulting from the considerations in this section are given in Table
\ref{tab:NExposure}; each of these exposures will result in a full camera read (6.4 Gby uncompressed).

We need 6 $\dspotProjector$ exposures to synthesize the full angular distribution of rays traversing the
system using minimal overlaps on M1; let as adopt a number 6$\OverfillFactor$ with a nominal value
$\OverfillFactor = 2$.  The
spots on M1 should be placed at a range of azimuthal angles, and that we will monitor the sensitivity
of derived calibration products to the positions of the spots, increasing $\OverfillFactor$ as appropriate.

Because we know $\qe_b(\lambda, \xb)$ we can take exposures \textit{without} any filter, so to cover
the full spectral range we need $70 \dlambdaWhite$ exposures.

We also need a spatial model of the sensitivity.  If all the spots had the same brightness we could use the
data taken to study the spectral response to model the (smoothly-varying) spatial structure.  As each of these
exposures includes $945 \nspotCCD$ spots this should be sufficient to map the expected smooth variation of the
monochromatic throughput (\eg we could use radial basis functions or expand to \c 45\th order Zernikes).  In
theory we only need perform this test at a single wavelength, but it seems prudent to check for
wavelength-dependent effects in the projector (\eg chromatic effects in the projector optics, diffraction
moving light outside our measurement apertures).

If the CCDs show spatial features with significant wavelength dependencies (\eg features in the
anti-reflection coatings) we may need to increase $\Noffset$ for at least some wavelengths. The current test
plan at BNL (\cite{GilmoreCCDTesting}) asks the vendors for QE measurements at 5 positions per CCD and this
will probably suffice.  We do not include an estimate of the cost of this possibility in Table
\ref{tab:NExposure}.

In fact, as discussed in section \ref{secProjector}, the illumination and sizes of all the holes in the mask
won't be identical and we will need to take a number of exposures with different mask offsets and then solve
for the intensity of each spot and the size of each hole.  A detailed study of the number of offsets,
$\Noffset$ needed is \TBD; we shall adopt a nominal value of $\NoffsetNominal$ (we can also reuse one of the
exposures from the spectral scans).  Because the study need not be repeated at more than one projector
position the total calibration time is insensitive to the value of $\Noffset$.

We also need to know the small-scale variability of the response, primarily due to pixel effects.  These are
not expected to show strong spectral features, and may be measured using the flat field screen at a relatively
coarse wavelength resolution.  It is not easy to think of pixel-scale
effects that evolve, but this data set may also be used to monitor the non-azimuthally symmetrical
evolution of the system.

What can evolve is structures on the filters (although dust motes are far smaller than the beam
size at the filter).  We can monitor contamination with a single flat field taken at the central wavelength,
but we allow for taking a small number of flat fields per filter.

These exposures may also be used to monitor the gain, although better measurements may be obtained using
cosmic rays in science and dark exposures.

While taking flats we may as well also take bias and dark measurements (although these could also be taken on
dark and stormy nights).

Finally, we need to know the fringe frames (especially in the y-band).  We care about fringing in two ways: it
produces small-scale structure in the sky, and it produces small-scale modulation in the throughput. It is
possible that the former will be best measured from the sky, but the latter should be measured directly. Once
again we can use unfiltered light, but need to scan at 1nm resolution. This is expensive.  In reality we can
probably relax this requirement.  In the blue there are very few bright lines to worry about, and the chips
are thicker in units of wavelength.  Because fringes cover many pixels we can probably use data with lower S/N
per pixel (recovering the fine structure from the regular flats);  together these approaches are worth
at least a factor of two, and probably five.

\subsubsection{Frequency of Measurements}

Please refer to Table \ref{tab:NExposure} for the exposure times.

Not all these measurements need to be made at a common cadence.  Because of the possibility of contamination
we need to measure the per-pixel with-filter flats frequently; we say `daily' in Table \ref{tab:NExposure} but
weekly would probably be sufficient (or possibly whenever we change filters).  These are the only
calibration exposures that need filters, and are therefore the only exposures that need pay the cost
of filter changes.

The cost of the no-filter flat fields is comparable, but monthly scans should be good enough
as it is hard to think of effects that vary even that fast.

The dark exposures are put in at a weekly cadence, but they are expected to vary only slowly if at all;
the bias exposures are cheap and monitor changes in the noise properties of the amplifiers so we may
as well take them daily.

The system throughput, like the no-filter flat fields, would be expected to vary only slowly, and we
propose that they be taken at the same cadence, monthly.  Note that we could take a subset more often.
Whenever we take spot-projector flats we should also measure the mask illumination, but this is relatively
cheap.

Finally, the fringes are not expected to vary as they are defined by the silicon; we conservatively
take a set of fringe frames every year, but expect that this could be relaxed.  Unfortunately doing
so does not diminish the needed calibration in the first year unless the camera team delivers the
required fringe data.

\subsection{The Filter Transmission}
\label{sec:MeasureFilters}

One option for measuring the filter transmissions as a function of position is to build a machine that
passes an f/1.2 beam through the filter and into a spectrophotometer; if the beam diameter is \c 10cm
this directly measures the transmission as seen by the camera at that point in the filter, and we only
need \c 50 exposures to fully characterise the system.

In the absence of such a machine,
the spot projector can be used to monitor or measure the filter throughput. The area of the filter illuminated
is about $2.1\% \nspotCCD$, and we need \c 2520 $\dspotProjector \overfillFactor \dlambdaFilter \nband
\fracBandwidth$ exposures to cover the full spectral range; 31.5 hours.  To explore all positions on the
filter would take 1470 hours with 10s flats.

If we assume that the non-filter part of the transmission has no azimuthal symmetry, then these estimates must
be increased by a factor of \c 45 and a full exploration would be prohibitively expensive.

It seems likely that we can use at least ten times as many spots without causing significant confusion due to
ghosting, in which case each pointing of the collimated exposure illuminates \c 20\% of the filter and we gain
a factor of 10, so around 150 hours of integration.  If we're willing to make stronger assumptions about the
spatial coherence of the evolution of the filters we can obviously use fewer exposures.  For example, if we
assume that the properties of the filter don't vary significantly on the scale of the 10cm disk illuminated by
M1, the 31.5 hour number quoted above is sufficient.

\endgroup                               % macros
